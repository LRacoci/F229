\documentclass[10pt,english]{article}
\usepackage[T1]{fontenc}
\usepackage[utf8]{inputenc}
\usepackage[a4paper]{geometry}
\geometry{verbose}
\usepackage{color}
\usepackage{amsmath}
\usepackage{amssymb}

\makeatletter
%\usepackage[brazilian]{babel}
\usepackage{amsfonts}\usepackage{caption}\usepackage{babel}
\usepackage{babel}
\usepackage{babel}
\usepackage{babel}
\makeatother
\usepackage{babel}

\begin{document}

\title{\vspace{50mm}
 \textbf{\textcolor{blue}{\Huge{Experimento 04 }}}\textbf{\textcolor{green}{\Huge{Máquina
de Atwood}}}}


\author{Beatriz Sechin Zazulla \hfill{}\textit{RA: 154779}\protect\\
 Guilherme Lucas da Silva\hfill{}\textit{RA: 155618}\protect\\
 Henrique Noronha Facioli \hfill{}\textit{RA: 157986}\protect\\
 Isadora Sophia \hfill{}\textit{RA: 158018}\protect\\
 Lucas Alves Racoci \hfill{}\textit{RA: 156331} }

\maketitle
\newpage{}


\section{Resumo}

Neste experimento, estudamos uma \emph{Máquina de Atwood}, um sistema
físico que consiste de: 
\begin{enumerate}
\item Um cilindro de latão funcionando como polia, ou seja com liberdade
de girar em torno de um eixo fixo; 
\item Um fio que será considerado:

\begin{enumerate}
\item Leve, isto é com massa irrelevante e 
\item Inestensível, isto é, inelástico. 
\end{enumerate}
\item Dois corpos: 1 e 2 , pendurados na polia por meio do fio anteriormente
citado, onde:

\begin{itemize}
\item O corpo 1 consiste de um sub-corpo de massa $\widetilde{m}_{1}$ e
mais $n_{1}$ de $5$ sub-corpos; 
\item O corpo 2 consiste de um sub-corpo de massa $\widetilde{m}_{2}$ e
mais $n_{2}$ de $5$ sub-corpos; 
\item $n_{1}$ e $n_{2}$ são tais que: $n_{1}+n_{2}=5$; 
\item As massas dos corpos 1 e 2 serão chamadas respectivamente de $m_{1}$
e $m_{2}$ 
\end{itemize}
\end{enumerate}
Sabemos que a diferença entre as massas dos dois corpos gera um torque
não nulo na polia, o que nos permite estudar seu Momento de Inercia
$I$, ou a aceleração da grávidade $g$ através da fórmula a seguir:

\[
\Delta m=\cfrac{2h}{gR^{2}}(I+MR^{2})\ensuremath{\cfrac{1}{t^{2}}}+\cfrac{\tau_{a}}{gR}
\]



\section{Objetivos}

Este experimento teve como objetivo principal o estudo da máquina
de Atwood e a determinação do momento de inérciada da polia e o torque
da força de atrito.


\section{Procedimento Experimental e Coleta de Dados}


\subsection{Materiais utilizados}

Na realização deste experimento foram utilizados os seguintes materiais: 
\begin{itemize}
\item Polia de latão com euixo; 
\item Barbante; 
\item Dois pesos de suspensão; 
\item Conjunto de discos mestálicos; 
\item Trena; 
\item Paquimetro; 
\item Balança de Precisão e 
\item Cronômetro 
\end{itemize}

\subsection{Procedimento}


\subsection{Dados Obtidos}


\section{Análise dos Resultados e Discussões}


\subsection{Linearização}


\subsection{Regressão linear}


\section{Conclusões}
\end{document}
